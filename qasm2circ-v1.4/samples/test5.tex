% 
% File:	  test5.qasm
% Date:	  22-Mar-04
% Author: I. Chuang <ichuang@mit.edu>
%
% Sample qasm input file - demonstrate arbitray qubit matrix ops
% 
% 	def	c-P,1,'\m{e^{i\alpha} & 0 \cr 0 & e^{-i\alpha}}'
% 	def	Ryt,0,'\m{\cos{\theta}&-\sin{\theta}\cr\sin{\theta}&\cos{\theta}}'
% 
% 	qubit	j0
% 	qubit	j1
% 
% 	c-P	j0,j1
% 	Ryt	j0
%  Time 01:
%    Gate 00 c-P(j0,j1)
%  Time 02:
%    Gate 01 Ryt(j0)

% Qubit circuit matrix:
%
% j0: gAA, gBA, n  
% j1: gAB, n  , n  

\documentclass[11pt]{article}
%
% File:   xyqcirc.tex
% Date:   14-Mar-04
% Author: I. Chuang <ichuang@mit.edu>
%
% Definitions for producing quantum circuits with XYPIC in latex
%
% $Log: xyqcirc.tex,v $
% Revision 1.17  2004/03/25 05:01:23  ike
% discard and slash
%
% Revision 1.16  2004/03/25 04:58:42  ike
% added discard, and variable width dmeter
%
% Revision 1.15  2004/03/24 23:43:33  ike
% \dmeter and \sq
%
% Revision 1.14  2004/03/24 20:29:40  ike
% added \t for swap
%
% Revision 1.13  2004/03/24 17:52:16  ike
% removed \w from \gspace
%
% Revision 1.12  2004/03/24 16:38:34  ike
% added small space before |0> for \z
%
% Revision 1.11  2004/03/24 16:23:11  ike
% added \z
%
% Revision 1.10  2004/03/24 16:19:11  ike
% added multiple qubit operations
%
% Revision 1.9  2004/03/24 03:03:44  ike
% typo
%
% Revision 1.8  2004/03/24 02:50:09  ike
% added qv
%
% Revision 1.7  2004/03/24 00:07:34  ike
% add \m matrix op
%
% Revision 1.6  2004/03/23 23:13:10  ike
% misc
%
% Revision 1.5  2004/03/23 23:12:42  ike
% works now
%
% Revision 1.4  2004/03/23 22:22:34  ike
% ifthen also failes - because xymatrix entries in \save...\restore
%
% Revision 1.3  2004/03/23 21:34:36  ike
% no q/c wire switching
%
% Revision 1.2  2004/03/23 21:25:29  ike
% classical qo quantum wire switching try
%
% Revision 1.1  2004/03/23 21:01:46  ike
% Initial revision
%

%%%%%%%%%%%%%%%%%%%%%%%%%%%%%%%%%%%%%%%%%%%%%%%%%%%%%%%%%%%%%%%%%%%%%%%%%%%%%
% preliminaries

\usepackage{graphicx}

\usepackage[frame,line,arrow,matrix,tips]{xy}	% all that is usually necessary
\CompilePrefix{xygui-}
\makeindex
\pagestyle{empty}

\setlength{\oddsidemargin}{-0.5in}	% 1.25in left margin 
\setlength{\evensidemargin}{-0.5in}	% 1.25in left margin (even pages)

\setlength{\topmargin}{0.0in}		% 1in top margin
\setlength{\textwidth}{6.25in}		% 6.0in text - 1.25in rt margin
\setlength{\textheight}{8.6in}		% Body ht for 1in margins
\addtolength{\topmargin}{-\headheight}	% No header, so compensate
\addtolength{\topmargin}{-\headsep}	% for header height and separation

\begin{document}

\thispagestyle{empty}

%%%%%%%%%%%%%%%%%%%%%%%%%%%%%%%%%%%%%%%%%%%%%%%%%%%%%%%%%%%%%%%%%%%%%%%%%%%%%
% wires

\def\w{\ar@{-}[l]}
\def\W{\ar@{=}[l]}

%%%%%%%%%%%%%%%%%%%%%%%%%%%%%%%%%%%%%%%%%%%%%%%%%%%%%%%%%%%%%%%%%%%%%%%%%%%%%
% labels

% simple label
\def\A#1{\save []="#1" \restore}

%%%%%%%%%%%%%%%%%%%%%%%%%%%%%%%%%%%%%%%%%%%%%%%%%%%%%%%%%%%%%%%%%%%%%%%%%%%%%
% single qubit operations

\def\op#1{*+[F]{\rule[-0.2ex]{0ex}{2.1ex}#1}}	% operator in box
\def\b{*={\bullet}}
\def\o{*={\oplus}}
\def\t{*={\times}}				% for swap gate
\def\sq{*=<6pt,6pt>[F]{}}			% square, for controlled-phase
\def\m#1{\left[\matrix{#1}\right]}		% matrix shortcut
\def\z{*+[]{\rule[-0.2ex]{0ex}{2.1ex}~|0\>}}	% re-init to |0>
\def\discard{*[]{\rule[-0.2ex]{0.75pt}{2.1ex}~}}	% vertical ``|''
\def\slash{*={/}}				% slash for wire bundles

%%%%%%%%%%%%%%%%%%%%%%%%%%%%%%%%%%%%%%%%%%%%%%%%%%%%%%%%%%%%%%%%%%%%%%%%%%%%%
% nop's

\def\N{*-{}\W}
\def\n{*-{}\w}

%%%%%%%%%%%%%%%%%%%%%%%%%%%%%%%%%%%%%%%%%%%%%%%%%%%%%%%%%%%%%%%%%%%%%%%%%%%%%
% misc definitions

\def\>{\rangle}
\def\<{\langle}
\def\ua{\uparrow}

% measurement box
\def\meter{*+[]{\put(-3,0){\includegraphics[scale=.5]{meter.epsf}}~~~~}%
		\ar@{-}[l]}

%%%%%%%%%%%%%%%%%%%%%%%%%%%%%%%%%%%%%%%%%%%%%%%%%%%%%%%%%%%%%%%%%%%%%%%%%%%%%
% qubit names (and also revert to qubit wires, vs, cbit wires)

\def\q#1{*+{\rule[-0.2ex]{0ex}{2.1ex}|#1\>}}
\def\qv#1#2{*+{\rule[-0.2ex]{0ex}{2.1ex}|#1\>=|#2\>}}
	
%%%%%%%%%%%%%%%%%%%%%%%%%%%%%%%%%%%%%%%%%%%%%%%%%%%%%%%%%%%%%%%%%%%%%%%%%%%%%
% multiple qubit gates

% utulity text box for figuring out width of things
\newbox{\sbox}

% empty space of width determined by the text argument
\def\gspace#1{*+{\rule[-0.2ex]{0ex}{2.1ex}%
	\setbox\sbox=\hbox{$#1$}%
	\hspace*{\wd\sbox}}}
	
% n-qubit operation #1=box label, #2=number of qubits (eg d=2 qubits, ddd=4)
\def\gnqubit#1#2{\gspace{#1}
		 \save [].[#2]!C="qq"*[F]\frm{}\restore
		 \save "qq"*[]{#1} \restore}

% two-qubit operation
\def\gtwo#1{\gnqubit{#1}{d}}

% two-qubit operation
\def\gthree#1{\gnqubit{#1}{dd}}

%%%%%%%%%%%%%%%%%%%%%%%%%%%%%%%%%%%%%%%%%%%%%%%%%%%%%%%%%%%%%%%%%%%%%%%%%%%%%
% ``D'' style measurement gate a-la-cleve, at Michael Nielsen's request

\def\dmeterwide#1#2{*{\xy <0pt,-8pt>;<0pt,8pt> **@{-};
		    <0pt,-8pt>;<#2,-8pt> **@{-} ;
		    <0pt, 8pt>;<#2, 8pt> **@{-} ;
		    <#2,0pt>-<5pt,0pt>*{#1} ;
		    <#2,0pt>*\cir<8pt>{r_l}\endxy}}

\def\dmeter#1{\dmeterwide{#1}{12pt}}

%%%%%%%%%%%%%%%%%%%%%%%%%%%%%%%%%%%%%%%%%%%%%%%%%%%%%%%%%%%%%%%%%%%%%%%%%%%%%


% definitions for the circuit elements

\def\gAA{\b\w\A{gAA}}
\def\gAB{\op{\m{e^{i\alpha} & 0 \cr 0 & e^{-i\alpha}}}\w\A{gAB}}
\def\gBA{\op{\m{\cos{\theta}&-\sin{\theta}\cr\sin{\theta}&\cos{\theta}}}\w\A{gBA}}

% definitions for bit labels and initial states

\def\bA{ \q{j_{0}}}
\def\bB{ \q{j_{1}}}

% The quantum circuit as an xymatrix

\xymatrix@R=5pt@C=10pt{
    \bA & \gAA &\gBA &\n  
\\  \bB & \gAB &\n   &\n  
%
% Vertical lines and other post-xymatrix latex
%
\ar@{-}"gAB";"gAA"
}

\end{document}
